\newcommand{\tekenbevoegd}{Albertus Steenbeeke}
\newcommand{\functie}{General Manager}
\newcommand{\onderneming}{BT\&T Software Development SRL}
\newcommand{\vestigingsplaats}{Boekarest}
\documentclass[10pt, a4paper]{article}
\usepackage{eurosym}
\usepackage[a4paper,top=2.0cm, bottom=1.0cm]{geometry}
\usepackage[utf8]{inputenc}
\usepackage[dutch]{babel}
\hyphenation{niet-sub-licen-si-eer-ba-re}
\renewcommand{\rmdefault}{ppl}
\linespread{1.2}
\setlength{\parindent}{0in}
\title{Licentieovereenkomst GOVI data-producten\vspace{-2ex}}
\begin{document}
\maketitle
\thispagestyle{empty}
\pagestyle{empty}
\textbf{Stichting OpenGeo}, gevestigd en kantoorhoudende te Leiden, rechtsgeldig vertegenwoordigd door \textbf{Stefan de Konink, voorzitter}, hierna te noemen ``Licentieverlener namens GOVI'';
\begin{center}
en\\
\end{center}
\ifdefined\onderneming
    \hyphenation{\tekenbevoegd}
    \textbf{\onderneming} te dezer zake rechtsgeldig vertegenwoordigd door \textbf{\tekenbevoegd, \functie}, hierna verder te noemen ``Licentienemer'';
\else
    \textbf{\tekenbevoegd}, hierna te noemen ``Licentienemer'';
\fi

\section*{Overwegende dat:}
\begin{enumerate}
\item Het project GOVI `Grenzeloze Openbaar Vervoer Informatie' door de Provincie Noord-Holland is opgezet ten behoeve van het verzorgen van dynamische reisinformatie voor en aan reizigers in het openbaar vervoer (verder te noemen: OV);
\item De Provincie Noord-Holland met diverse andere decentrale overheden (verder te noemen: Openbaar Vervoer Autoriteiten) in 2011 de ``Bestuursovereenkomst GOVI'' heeft gesloten waarin afspraken zijn gemaakt over samenwerking en financiering met betrekking tot een OV-reisinformatiesysteem;
\item De Provincie Noord-Holland namens de deelnemers aan de Bestuursovereenkomst GOVI het project GOVI uitvoert en daarom als Licentieverlener optreedt;
\item Licentieverlener zich met middelen van de Openbaar Vervoer Autoriteiten inzet voor de stimulering van het optimaal informeren van de reiziger op het gebied van (dynamische)reisinformatie door het digitaal ter beschikking stellen van de benodigde data aan bedrijven en kennisinstellingen;
\item De OV-bedrijven inhoudelijk verantwoordelijk zijn voor de kwaliteit van de aan Licentieverlener  geleverde data en deze wordt omgezet naar GOVI dataproducten ten behoeve van Licentienemer;
\item Licentienemer de gegevens van Licentieverlener wenst te gebruiken om reisinformatie diensten mee te ontwikkelen;
\item Licentieverlener aan Licentienemer het recht wenst te verlenen de GOVI dataproducten te ontvangen en onder voorwaarden te distribueren naar derden.
\end{enumerate}

\newpage

\section{Begripsbepalingen}
\begin{itemize}
\item Backwards compatibility: toekomstige wijzigingen hebben geen invloed op bestaande implementaties.
\item CC0 licentie: licentie van Creative Commons Corporation welke tot doel heeft informatie vrij van gebruiksvoorwaarden te distribueren, waarbij ketenaansprakelijkheid wordt uitgesloten.
\item GOVI dataproducten: openbaar vervoer informatie, afkomstig van meerdere OV Bedrijven, welke op geïntegreerde wijze ter beschikking wordt gesteld.
\item OV Bedrijven: vervoerders die openbaar vervoer verrichten op basis van een concessie.
\end{itemize}

\section{Omvang licentie}
\begin{enumerate}
\item Deze licentieovereenkomst heeft alleen betrekking op dynamische reisinformatie (o.a. KV8 en KV55). Planningsinformatie (o.a. KV1 en KV7) valt onder de CC0 licentie, zie bijlage A;
\item Voor alle data die Licentienemer ontvangt heeft Licentieverlener toestemming van de desbetreffende Openbaar Vervoer Autoriteiten en is deze overeenkomst van kracht;
\item Voor data van gespecificeerde concessies geldt de CC0 licentie. De betreffende uitzonderingen zijn opgenomen in bijlage B. Deze bijlage kan gedurende de looptijd van deze overeenkomst uitgebreid worden.
\end{enumerate}

\section{Licentie}
\begin{enumerate}
\item Licentieverlener verleent aan de Licentienemer het niet-exclusieve recht om voor de duur van deze overeenkomst GOVI dataproducten (of delen ervan) te gebruiken om reisinformatie diensten mee te ontwikkelen en/of aan derden te distribueren die tot doel hebben reisinformatie diensten te ontwikkelen.
\item Licentieverlener levert de data uitsluitend in de vorm van standaard koppelvlakken zoals gespecificeerd in bijlage C.
\end{enumerate}

\section{Voorwaarden aan levering GOVI data-producten}
\begin{enumerate}
\item Licentieverlener stelt de volgende voorwaarden bij het in artikel 3 genoemde recht:
    \begin{enumerate}
    \item Licentienemer gebruikt de GOVI data-producten uitsluitend voor het in artikel 3 genoemde doeleinde.
    \item Licentienemer is verantwoordelijk voor technische en fysieke interne beveiliging van de GOVI data-producten, zodat onbevoegd gebruik wordt voorkomen.
    \item Licentienemer stelt de Licentieverlener te allen tijde technisch en functioneel is staat te controleren of de voorwaarden in dit artikel door Licentienemer worden nageleefd.
    \item Licentienemer zal alles in het werk stellen om de systemen van Licentieverlener niet te verstoren (e.g. door overbelasting). Indien verstoring optreedt, gaat het primaire proces van Licentieverlener voor en kan de koppeling met Licentienemer worden verbroken. 
\newpage
    \item Licentienemer zal verwijzen naar Licentieverlener in publicaties over de door hem geleverde reisinformatieproducten waarin de GOVI data-producten worden gebruikt.
    \item Licentienemer staat er voor in dat derden die de GOVI data-producten ontvangen van  Licentienemer zich conformeren aan alle voorwaarden van deze overeenkomst.
    \end{enumerate}
\item Licentieverlener is gerechtigd de licentievoorwaarden tussentijds aan te passen, bijvoorbeeld met betrekking tot het vrijgeven van nieuwe concessies onder de CC0 licentie. 
\end{enumerate}

\section{Verplichtingen van Licentieverlener}
\begin{enumerate}
\item Licentieverlener draagt zorg voor levering van GOVI data-producten aan Licentienemer en zal al bekende leveringsonderbrekingen aan Licentienemer melden.
\item Onderbrekingen van datalevering kunnen door Licentienemer bij Licentieverlener gemeld worden. Licentieverlener spant zich naar beste kunnen in om gemelde leveringsproblemen spoedig te herstellen.
\item Data problemen van kwalitatief-inhoudelijke aard zoals verkeerde tijden en verkeerde haltenamen kunnen door Licentienemer gemeld worden aan Licentieverlener. Licentieverlener bepaalt of en zo ja hoe deze problemen worden behandeld.
\item Licentieverlener streeft naar backwards compatibilaty, eventueel noodzakelijke wijzigingen worden zes maanden voor ingang door Licentieverlener gecommuniceerd.
\end{enumerate}


\section{Vergoeding en betaling}
\begin{enumerate}
\item Voor het aan Licentienemer verleende recht, zoals bedoeld in artikel 3 van deze overeenkomst, is Licentienemer geen vergoeding verschuldigd. 
\item Het is Licentienemer toegestaan om aan derden een vergoeding te vragen voor het leveren van GOVI dataproducten, met een maximum van \euro 1.000,- per jaar per derde. Eventuele vergoeding is bedoeld ter dekking van benodigde kosten voor distributie van GOVI data-producten.
\end{enumerate}

\section{Eigendomsrechten en aansprakelijkheid}
\begin{enumerate}
\item Licentieverlener staat er jegens Licentienemer voor in dat hij alle rechten op de GOVI data-producten bezit en dat hij bevoegd is de rechten genoemd in deze overeenkomst aan Licentienemer te verlenen.
\item Deze overeenkomst beoogt niet de overdracht van intellectuele eigendomsrechten op de GOVI data-producten. Alle intellectuele eigendomsrechten ten aanzien van de GOVI data, of dit nu auteursrechten, databankrechten en of andere (intellectuele eigendoms) rechten zijn, blijven bij de eigenaar daarvan.
\item Aan eventuele onderbrekingen van data levering kunnen geen rechten worden ontleend door Licentienemer.
\item Licentieverlener kan niet aansprakelijk worden gesteld voor geleden schade in verband met het gebruik van de GOVI data-producten.
\end{enumerate}
\newpage
\section{Duur en beëindiging}
\begin{enumerate}
\item Deze overeenkomst vangt aan op de datum van laatste ondertekening en loopt tot en met \textbf{31 december 2015}. Licentieverlener kan de overeenkomst eenzijdig verlengen.
\item Indien Licentienemer niet voldoet aan de in artikel 4 gestelde voorwaarden, of een of meerdere afnemers van Licentienemer niet voldoen aan de licentievoorwaarden Service provider, zulks ter beoordeling van Licentieverlener, kan de datalevering per direct worden opgeschort totdat Licentienemer ten genoegen van Licentieverlener kan aantonen dat aan alle voorwaarden wordt voldaan.
\item De Licentieverlener is te allen tijde gerechtigd deze overeenkomst door middel van een aangetekend schrijven en een opzegtermijn van zeven kalenderdagen tussentijds te beëin-digen indien:
\begin{enumerate}
\item Licentienemer na schriftelijke ingebrekestelling niet binnen zeven kalender dagen aan alle voorwaarden, zoals gesteld in artikel 4, voldoet;
\item Bij herhaalde verstoring als bedoeld in artikel 4, lid 1, sub d van deze overeenkomst, indien Licentienemer niet kan aantonen hoe deze verstoringen in de toekomst zullen worden voorkomen, zulks ter beoordeling door Licentieverlener.
\end{enumerate}
\end{enumerate}

\section{Geschillen en toepasselijk recht}
\begin{enumerate}
\item Op deze overeenkomst is Nederlands recht van toepassing.
\item Ieder geschil tussen partijen ter zake van deze overeenkomst zal, indien minnelijke oplossing van dit geschil niet mogelijk is gebleken, bij uitsluiting worden voorgelegd aan de daartoe bevoegde rechter in het arrondissement Haarlem, tenzij partijen alsnog arbitrage of bindend advies overeenkomen.
\item Afwijkende bedingen, wijzigingen van en/of aanvullingen op deze overeenkomst gelden slechts indien en voor zover deze tussen de Licentieverlener en Licentienemer uitdrukkelijk schriftelijk zijn overeengekomen.
\item Indien een bepaling van deze overeenkomst nietig is of vernietigd wordt, blijven de overige bepalingen volledig van kracht. Licentieverlener en Licentienemer zullen dan in overleg treden teneinde een nieuwe bepaling ter vervanging van de nietige of vernietigde bepaling overeen te komen, waarbij zoveel mogelijk met het doel en strekking van de nietige of vernietigde bepaling rekening zal worden gehouden.
\item De bijlagen bij deze overeenkomst maken onlosmakelijk deel uit van deze overeenkomst.
\end{enumerate}

\subsubsection*{}
In tweevoud ondertekend op \today, te \vestigingsplaats.\\
\\
\\
\begin{tabular}{l p{3.3cm} l}
Namens Licentienemer: & & Namens Licentieverlener GOVI:\\
\\
\vspace{6ex} \\
\\
\tekenbevoegd & & Stefan de Konink \\
\ifdefined\onderneming\onderneming\fi & & Stichting OpenGeo \\
\end{tabular}

\newpage
\appendix
\section{CC0 1.0 Universeel}
\textit{Creative Commons Corporation is geen advocatenpraktijk en verleent geen juridische diensten. De verspreiding van deze licentie roept geen juridische relatie met Creative Commons in het leven. Creative Commons verspreidt deze informatie 'as-is'. Creative Commons staat niet in voor de inhoud van de verstrekte informatie en sluit alle aansprakelijkheid uit voor enigerlei schade voortvloeiend uit het gebruik van deze informatie indien en voorzover de wet niet anders bepaald.}

\subsection*{Doelstelling}
In het recht van de meeste landen wordt het exclusieve auteursrecht en de naburige rechten (zoals hieronder omschreven) automatisch toegekend aan de maker en de opeenvolgende rechthebbende(n) (hierna gezamenlijk en afzonderlijk aangeduid als ``rechthebbende'') op een oorspronkelijk auteursrechtelijk beschermd werk en/of database (hierna aangeduid als een ``Werk'').
Sommige rechthebbenden wensen afstand te doen van hun rechten op een werk om zo een bijdrage te leveren aan een gemeenschap van vrij toegankelijke creatieve, culturele en wetenschappelijke werken (``Commons''), dat iedereen kan bewerken, aanpassen, opnemen in andere werken, hergebruiken of herdistribueren, ongeacht het doel, inclusief (maar niet beperkt tot) voor commerciële doeleinden, zonder bang te hoeven zijn dat deze rechten worden geschonden. De rechthebbenden kunnen bijvoorbeeld hun werk beschikbaar stellen aan de Commons om het idee van een vrije cultuur en de productie van creatieve, culturele en wetenschappelijke werken te promoten, of om grotere bekendheid te geven aan hun werk of de verspreiding ervan bevorderen, deels door het gebruik en de inspanningen van anderen.
Voor deze en/of andere doeleinden en redenen, en zonder aanspraak te maken op een aanvullende tegenprestatie of vergoeding, kiest degene die CC0 in verband brengt met een Werk (de ``Bekrachtiger'') er vrijwillig voor, voor zover hij of zij houder is van het auteursrecht en naburige rechten op het Werk, om CC0 op het Werk van toepassing te verklaren en het Werk in het publieke domein te brengen, onder de voorwaarden van CC0, en met inachtneming van zijn of haar auteursrecht en naburige rechten op het Werk en van de betekenis en desbetreffende juridische gevolgen van CC0 voor die rechten.

\subsection{Auteursrecht en naburige rechten}
Op een Werk dat op grond van CC0 beschikbaar is gesteld, kunnen auteursrecht en aanverwante of naburige rechten rusten (``Auteursrecht en Naburige Rechten''). Auteursrecht en naburige rechten omvatten maar zijn niet beperkt tot:
\begin{enumerate}
\item het recht om een Werk te verveelvoudigen, bewerken, distribueren, uit te voeren, tentoon te spreiden, communiceren en vertalen;
\item de door de oorspronkelijke auteur(s) en/of uitvoerder(s) behouden morele rechten;
\item portretrecht en privacy rechten die verband houden met een in een Werk opgenomen beeltenis of gelijkenis van een persoon;
\item rechten die bescherming bieden tegen oneerlijke concurrentie ten aanzien van een werk, met inachtneming van de in onderstaand artikel 4, lid a, opgenomen beperkingen;
\item rechten die bescherming bieden tegen het opvragen, verspreiden, gebruiken en hergebruiken van in een Werk opgenomen gegevens;
\item databankrechten (zoals die welke voortvloeien uit Richtlijn 96/9/EG van het Europees Parlement en de Raad van 11 maart 1996 betreffende de rechtsbescherming van databanken en uit nationaal recht krachtens de implementatie van de richtlijn daarin, inclusief de gewijzigde of opeenvolgende versie van die richtlijn); 
\item overige vergelijkbare, gelijkwaardige, of overeenkomende rechten waar ook ter wereld, op basis van het toepasselijke recht of verdrag en de implementatie daarvan in nationaal recht.
\end{enumerate}

\subsection{Afstandsverklaring}
Voor zover dit is toegestaan onder en niet in strijd is met het toepasselijke recht, doet Bekrachtiger hierbij uitdrukkelijk, volledig, permanent, onherroepelijk en onvoorwaardelijk afstand van al zijn/haar auteursrechten en naburige rechten en van alle daarmee verband houdende vorderingen en gerechtelijke procedures, hetzij bekend of onbekend (en zowel bestaande als toekomstige vorderingen en gerechtelijke procedures), op het Werk en geeft deze op (i) in alle rechtsgebieden ter wereld, (ii) voor de maximale duur toegestaan onder het toepasselijke recht of verdrag (inclusief toekomstige verlengingen daarvan), (iii) in elk bestaand of in de toekomst te ontwikkelen medium en ongeacht het aantal kopieën, en (iv) ongeacht het doeleinde, daaronder begrepen maar niet beperkt tot commerciële, reclame of promotionele doeleinden (de ``Afstandsverklaring'').
Bekrachtiger doet deze afstandsverklaring ten behoeve van het publieke domein en ten nadele van zijn/haar erfgenamen en rechtsopvolgers, met de volledige intentie deze afstandsverklaring niet te herroepen, ontbinden, annuleren of beëindigen, en verklaart dat met betrekking daartoe geen rechtsvordering of andere gerechtelijke actie zal worden ingesteld die het ongestoord genot van het Werk door het publiek, zoals overwogen in de expliciete verklaring doelstelling van Bekrachtiger, zou kunnen verstoren.

\subsection{Publieke licentie}
Als enig deel van de afstandsverklaring onder het toepasselijke recht, om wat voor reden ook nietig of niet rechtsgeldig worden verklaard, dan blijft de afstandsverklaring overigens zoveel mogelijk van kracht, met inachtneming van de expliciete doelstelling van Bekrachtiger.
Daarnaast verleent Bekrachtiger, indien het bovenstaande met betrekking tot de afstandsverklaring wordt verklaard, elke getroffen persoon een royaltyvrije, niet-overdraagbare, niet-sublicensi-eerbare, niet-exclusieve, onherroepelijke en onvoorwaardelijke licentie om het auteursrecht en de naburige rechten van Bekrachtiger op het Werk uit te oefenen (i) in alle rechtsgebieden ter wereld, (ii) voor de maximale duur toegestaan krachtens het toepasselijke recht of verdrag (inclusief toekomstige verlengingen daarvan), (iii) in elk bestaand of in de toekomst te ontwikkelen medium en ongeacht het aantal kopieën, en (iv) ongeacht het doeleinde, inclusief zonder enige beperking commerciële, reclame of promotionele doeleinden (de ``Licentie''). De licentie wordt geacht van kracht te zijn met ingang van de datum waarop Bekrachtiger CC0 van toepassing heeft verklaard op het Werk.
Als enig deel van de licentie onder het toepasselijke recht om wat voor reden ook nietig of niet rechtsgeldig wordt verklaard, dan tast deze gedeeltelijke nietigheid of niet-rechtsgeldigheid de rest van de licentie niet aan en in dat geval bevestigt Bekrachtiger hierbij dat hij/zij (i) zijn/haar rechten op het resterende auteursrecht en naburige rechten op het Werk niet zal uitoefenen, en (ii) geen verwante rechtsvorderingen en gerechtelijke procedures met betrekking tot het Werk zal instellen, in beide gevallen in afwijking van de uitdrukkelijke verklaring doelstelling van Bekrachtiger.

\subsection{Beperkingen en afwijzing van verantwoordelijkheid}
\begin{enumerate}
\item Van geen enkel handelsmerk of octrooi van Bekrachtiger wordt afstand gedaan als gevolg van dit document, noch worden deze opgegeven, in licentie gegeven of anderszins aangetast.
\item Bekrachtiger biedt het Werk aan op ``as-is'' basis en doet geen enkele uitspraak en verstrekt geen enkele garantie met betrekking tot het Werk, expliciet, impliciet, wettelijk of anderszins, waaronder begrepen maar niet beperkt tot garantie van eigendom, verkoopbaarheid, geschiktheid voor een bepaald doel, niet-schending, of de afwezigheid van verborgen of andere gebreken, juistheid, of de aan- of afwezigheid van fouten, ongeacht of deze ontdekt kunnen worden, een en ander voor zover onder het toepasselijke recht is toegestaan.
\item Bekrachtiger wijst de verantwoordelijkheid af voor het vrijmaken van rechten van derden die op het werk of het gebruik daarvan kunnen berusten, zoals maar niet beperkt tot het auteursrecht en naburige rechten van derden op het Werk. Bekrachtiger is evenmin verantwoordelijk voor het verkrijgen van de voorgeschreven toestemmingen, vergunningen of andere rechten die vereist zijn voor het gebruik van het Werk.
\item Bekrachtiger erkent en verklaart dat Creative Commons geen partij is bij dit document en dat op haar geen enkele verplichting of plicht rust met betrekking tot deze CC0-verklaring of het gebruik van het Werk.
\end{enumerate}

\section{Uitzonderingen}
Voor de data van de hieronder genoemde concessies zijn de bepalingen in deze licentieovereenkomst ten aanzien van het gebruik van de data niet van kracht, maar geldt de CC0 licentie.

\subsection*{Stadsregio Amsterdam}
\begin{itemize}
\item Stadsvervoer Amsterdam
\item Waterland
\item Amstelland Meerlanden (incl. Zuidtangent)
\item Zaanstreek
\end{itemize}

\subsection*{Stadsgewest Haaglanden}
\begin{itemize}
\item Haaglanden agglomeratie Den Haag
\item Stadsbus Den Haag
\item Rail Haaglanden
\item Streekvervoer Haaglanden
\end{itemize}

\section{Techische Specificaties}
Technische specificaties van KV7/8turbo en KV55 kunnen op aanvraag beschikbaar worden gesteld. De set met koppelvlakken kan uitgebreid worden.

\end{document}
